%%%%%%%%%%%%%%%%%%%%%%%%%%%%%%%%%%%%%%%%%
% Medium Length Professional CV
% LaTeX Template
% Version 2.0 (8/5/13)
%
% This template has been downloaded from:
% http://www.LaTeXTemplates.com
%
% Original author:
% Rishi Shah
%
% Important note:
% This template requires the resume.cls
% file to be in the same directory as the
% .tex file. The resume.cls file provides
% the resume style used for structuring the
% document.
%%%%%%%%%%%%%%%%%%%%%%%%%%%%%%%%%%%%%%%%%

%-------------------------------------------
% PACKAGES AND OTHER DOCUMENT CONFIGURATIONS
%-------------------------------------------

\documentclass{resume} % Use the custom resume.cls style

\usepackage[left=0.75in,top=0.6in,right=0.75in,bottom=0.6in]{geometry} % Document margins
\usepackage{changepage}
\usepackage{hyperref}
\hypersetup{
    colorlinks=true,
    linkcolor=blue,
    filecolor=magenta,
    urlcolor=cyan,
    pdftitle={Overleaf Example},
    pdfpagemode=FullScreen,
}
\newcommand{\tab}[1]{\hspace{.2667\textwidth}\rlap{#1}}
\newcommand{\itab}[1]{\hspace{0em}\rlap{#1}}
\name{Joseph Gurr} % Your name
% \address{123 Pleasant Lane \\ City, State 12345} % Your address
% \address{123 Pleasant Lane \\ City, State 12345} % Your secondary address (optional)
\address{joseph.gurr@outlook.com} % Your phone number and email

\begin{document}

This document is created with the \TeX \ typesetting language and kept under version control. You can find the most recent version \href{https://github.com/joegurr/resume/blob/main/resume.pdf}{here}.

%------------------
% EDUCATION SECTION
%------------------

\begin{rSection}{Education}
    {\bf University of Newcastle, Australia} \hfill {\em 2014-2019}
    \\ B. Mathematics (Pure Mathematics) \hfill {\em GPA 6.96/7, WAM 91.39/100}
\end{rSection}

%------------------------
% CERTIFICATES SECTION
%------------------------

\begin{rSection}{Professional Certifications}
    \item Apache Druid Basics \hfill {\em expires Dec 2023}
\end{rSection}

%------------------------
% WORK EXPERIENCE SECTION
%------------------------

\begin{rSection}{Work Experience}

    \begin{rSubsection}{Sessional Acadmeic at The University of Newcastle}{\em Feburary 2022 - present}{}{}
        I run mathematics tutorials for first year science and engineering students. The content covers advanced calculus, linear algebra, and differential equations.
    \end{rSubsection}

    {\bf SwitchDin}

    SwitchDin is a energy startup that is trying to transform the world's energy grids to be highly
    distributed and renewable.

    \begin{adjustwidth}{2em}{}
        \begin{rSubsection}{Data Science Engineer}{\em July 2021 - Janurary 2022}{}{}
            \item My team developed distributed optimisation procedures that run on edge devices (Raspberry Pi) to control
            distributed energy resources. This involved having to run forecasts in real time, solve a convex optimisation problem, process the output, and
            then have the control implemented on an external device. This was done in Python.
        \end{rSubsection}

        \begin{rSubsection}{Data Services Team Lead (acting)}{\em January 2021 - July 2021}{}{}
            \item In the absence of a leader for the newly formed Data Services team I stepped up to the
            challenge and stood in as the temporary team leader. I lead a team of four data science
            professionals. This role included software development, internal and external stakeholder
            management, and management of my team members. Some of my responsibilities included providing
            and managing a performant and cost effective time series database, providing and operating an
            OLAP site used by our external clients, developing forecasting algorithms and collaborating
            with other teams on convex optimisation problems, developing market based decision making
            technologies, and more typical reporting functionality.
        \end{rSubsection}

        \begin{rSubsection}{Web Services Engineer}{\em May 2020 - January 2021}{}{}
            \item I built web and mobile applications using Angular and Ionic.
            I worked closely with our back-end teams, designers, and external customers
            to deliver the best product.
        \end{rSubsection}
    \end{adjustwidth}

    {\bf Heffron}

    Heffron provides administration, education, and other services to assist clients with their
    SMSFs (Self Managed Superannuation Funds). They are specialists in this field.

    \begin{adjustwidth}{2em}{}
        \begin{rSubsection}{}{\em November 2016 - November 2019}{}{}
            \item
            I began at Heffron in a data entry role. I quickly saw ways to automate many parts of my job.
            I taught myself how to write simple programs before taking some software engineering courses at uni.
            In my first year I was responsible for refactoring a critical algorithm that determines payable
            tax for certain SMSFs. This was done in Symfony (a php framework).
            I also worked on front end wizards that were written in KnockoutJS.

            From here I worked on how to make certain parts of the organisation more efficient. I helped increase
            to use of DocuSign by writing extensive internal documentation.
            I worked on assessing the regular expression engine used in one of the major pieces of software that
            the accountants used (Class).

            From here I worked on the massive project of transitioning our technology stack from a Symfony and KnockoutJS stack to a
            serverless architecture using AWS Lambdas (using Typescript). I used DynamoDB and IAM as a part of this. During this
            time I also worked on a customer facing CMS that our marketing team would ultimately manage.
        \end{rSubsection}
    \end{adjustwidth}

    {\bf Mathematics Undergraduate Summer Vacation Scholarship}
    \hfill {\em November 2015 - January 2016} \\
    I worked to develop a greater understanding of linear representations of k-regular
    sequences and Stern's diatomic sequence under the guidance of Associate Professor Michael Coons Jr.

        {\bf Private tutoring} \hfill {\em January 2012 - present} \\
    I have extensive experience tutoring secondary and tertiary students in mathematics,
    physics, chemistry, and statistics.
\end{rSection}

%---------------------
% TECHNOLOGIES SECTION
%---------------------

\begin{rSection}{Technologies/Programming Languages}
    \item {\em * I have worked with this in a professional capacity in a non-trivial way}
    \item {\bf Front-End} HTML* (plus a variety of HTML templating languages like razor,
    blade, and jinja2), CSS* (SASS*, SCSS*), Javascript*/Typescript* (Angular*, Ionic*)
    \item {\bf Back-End} Python* (Django*, Flask), NodeJS*, Java
    \item {\bf Data} RDBMS (Postgres*, MySQL*, MariaDB, SQLite), NoSQL (DynamoDB*, MongoDB),
    \\ Apache Druid*, Apache Superset*, RabbitMQ*, Kafka*, Redis*, Pandas*/Numpy*/other Python data packages*, R
    \item {\bf Operating Systems} Ubuntu* (16.04, 18.04*, 20.04*), MacOS X, Windows 10
    \item {\bf Other} git* (GitHub*, GitLab*, Bitbucket*), Docker*, AWS* (mainly S3*, Lambda*, APIGateway*, CloudFormation*), \TeX\ (\LaTeX),
    Atlassian Suite* (JIRA*, Confluence*)
    \item Other technologies/programming languages I have used in a limited capacity include:
    Ansible, Vue, React, LISP, PHP, C, C\texttt{++}, SPSS, MATLAB, Haskell, Docker Swarm, and Kubernetes
    \item I am experienced in CI/CD, TDD, and Agile
\end{rSection}

%------------------------
% ACHIEVEMENTS SECTION
%------------------------

\begin{rSection}{Achievements}
    \item Awarded the New Columbo Plan Mobility Scholarship to participate in \hfill {\em 2018}
    \\ the Reimagining India program
    \item Placed on the Faculty of Science Commendation List \hfill {\em 2015, 2016, 2018}
    \item Awarded the School of Mathematics and Physical Sciences Vacation Scholarship
    \hfill {\em 2015}
    \item Awarded the Shaping Futures Scholarship \hfill {\em 2015} \
    \item Awarded the Vice Chancellor's Scholarship for Academic Excellence in Year 12 \hfill {\em 2015}
\end{rSection}

\end{document}
